% Преамбула: настройки по ГОСТ 7.32-2017
% Компилятор: XeLaTeX

\documentclass[14pt,a4paper]{extreport}

% --- Шрифты (XeLaTeX) ---
\usepackage{fontspec}
\setmainfont{Liberation Serif} % метрически совместим с Times New Roman
\setsansfont{Liberation Sans}
\setmonofont{Liberation Mono}

% --- Язык ---
\usepackage[russian]{babel}

% --- Геометрия страницы (ГОСТ 7.32: 30/15/20/20 мм) ---
\usepackage[
  left=30mm,
  right=15mm,
  top=20mm,
  bottom=20mm
]{geometry}

% --- Межстрочный интервал 1.5 ---
\usepackage{setspace}
\onehalfspacing

% --- Абзацный отступ 1.25 см ---
\usepackage{indentfirst}
\setlength{\parindent}{1.25cm}

% --- Нумерация страниц внизу по центру ---
\usepackage{fancyhdr}
\pagestyle{fancy}
\fancyhf{}
\fancyfoot[C]{\thepage}
\renewcommand{\headrulewidth}{0pt}

% --- Заголовки (ГОСТ) ---
\usepackage{titlesec}

% Глава: ПРОПИСНЫМИ, по центру, жирным
\titleformat{\chapter}[block]
  {\normalfont\bfseries\large\centering}
  {\thechapter\quad}
  {0pt}
  {\MakeUppercase}
\titlespacing*{\chapter}{0pt}{-20pt}{12pt}

% Параграф: жирным, с абзацного отступа
\titleformat{\section}
  {\normalfont\bfseries\normalsize}
  {\thesection\quad}
  {0pt}
  {}
\titlespacing*{\section}{\parindent}{12pt}{6pt}

\titleformat{\subsection}
  {\normalfont\bfseries\normalsize}
  {\thesubsection\quad}
  {0pt}
  {}
\titlespacing*{\subsection}{\parindent}{12pt}{6pt}

% --- Содержание ---
\usepackage{tocloft}
\renewcommand{\cfttoctitlefont}{\hfill\bfseries\large\MakeUppercase}
\renewcommand{\cftaftertoctitle}{\hfill}
\renewcommand{\cftchapfont}{\normalfont}
\renewcommand{\cftchappagefont}{\normalfont}
\renewcommand{\cftchapleader}{\cftdotfill{\cftdotsep}}
\setlength{\cftbeforechapskip}{0pt}
\setcounter{tocdepth}{1} % только главы и секции (1, 1.1), без 1.1.1

% --- Рисунки и таблицы ---
\usepackage{graphicx}
\usepackage{float}
\usepackage{caption}
\captionsetup{
  labelsep=endash,
  justification=centering,
  font={normalsize,onehalfspacing}
}
\captionsetup[table]{
  justification=raggedleft,
  singlelinecheck=false
}

% Нумерация рисунков и таблиц в пределах главы
\usepackage{chngcntr}
\counterwithin{figure}{chapter}
\counterwithin{table}{chapter}

% --- Таблицы ---
\usepackage{longtable}
\usepackage{array}
\usepackage{booktabs}
\usepackage{tabularx}
\usepackage{multirow}

% --- Листинги кода ---
\usepackage{listings}
\lstset{
  basicstyle=\small\ttfamily,
  breaklines=true,
  frame=single,
  numbers=left,
  numberstyle=\tiny,
  tabsize=2,
  inputencoding=utf8,
  extendedchars=true
}

% --- Математика ---
\usepackage{amsmath}
\usepackage{amssymb}

% --- Гиперссылки ---
\usepackage{hyperref}
\hypersetup{
  colorlinks=true,
  linkcolor=black,
  citecolor=black,
  urlcolor=blue,
  hidelinks
}

% --- Библиография (ГОСТ 7.1) ---
\usepackage[
  backend=biber,
  style=gost-numeric,
  sorting=none,
  language=auto,
  autolang=other
]{biblatex}
\addbibresource{references.bib}

% --- Приложения ---
\usepackage[title]{appendix}

% --- Перечисления ---
\usepackage{enumitem}
\setlist{nosep, leftmargin=\parindent}

% --- Подавление сирот в заголовках ---
\widowpenalty=10000
\clubpenalty=10000
