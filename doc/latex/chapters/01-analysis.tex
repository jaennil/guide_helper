\chapter{Анализ предметной области и проектирование}

\section{Анализ предметной области цифрового туризма}

Современная туристическая индустрия переживает период активной цифровой трансформации. По данным Всемирной туристической организации (UNWTO), число международных туристических поездок к 2024 году превысило 1,3~млрд в год, при этом более 70\% путешественников используют цифровые инструменты при планировании и во время поездок~\cite{unwto2024}.

Особое место в этом процессе занимает сегмент активного туризма~--- пешего, горного, велосипедного. Рост интереса к экологичному и активному отдыху стимулирует спрос на специализированные цифровые инструменты для планирования маршрутов, навигации и обмена опытом.

Ключевым элементом туристического маршрута является визуальный контент~--- фотографии достопримечательностей, живописных видов, сложных участков, мест стоянок. Гиды, инструкторы и организаторы походов регулярно фотографируют ключевые точки маршрутов и хотят:
\begin{enumerate}
  \item разместить фотографии на интерактивной карте, привязав их к конкретным точкам маршрута;
  \item построить маршрут между этими точками с расчётом расстояния, набора высоты и времени в пути;
  \item поделиться готовым интерактивным маршрутом по ссылке, чтобы клиенты и участники похода могли открыть карту и просмотреть фотографии в контексте маршрута.
\end{enumerate}

Однако существующие решения не удовлетворяют этим потребностям в полной мере:
\begin{itemize}
  \item фотографии отправляются в мессенджеры и социальные сети без привязки к карте~--- теряется географический контекст;
  \item PDF-отчёты с вставленными фотографиями и скриншотами карт не интерактивны и не масштабируемы;
  \item специализированные сервисы (Wikiloc, Google My Maps) имеют существенные ограничения, рассмотренные в разделе~\ref{sec:competitors}.
\end{itemize}

Актуальность разработки информационной системы для создания туристических маршрутов с геопривязкой фотографий обусловлена:
\begin{itemize}
  \item ростом популярности активного туризма и экотуризма;
  \item увеличением числа самостоятельных путешественников;
  \item потребностью гидов в цифровых инструментах визуализации маршрутов с мультимедийным контентом;
  \item отсутствием открытого и бесплатного решения, объединяющего построение маршрутов, привязку фотографий и интерактивный обмен.
\end{itemize}


\section{Обзор и сравнительный анализ конкурентов}
\label{sec:competitors}

Для обоснования необходимости разработки собственной системы был проведён сравнительный анализ существующих решений для создания и обмена туристическими маршрутами с фотографиями.

\textbf{Wikiloc}~--- крупнейший международный сервис туристических GPS-треков с аудиторией более 13~млн пользователей~\cite{wikiloc}. Позволяет записывать GPS-треки во время похода и загружать фотографии, которые привязываются к треку автоматически по временной метке EXIF. Основные ограничения: фотографии нельзя привязать вручную к произвольным точкам маршрута; отсутствует интерактивный шаринг с фото-попапами; freemium-модель~--- навигация и оффлайн-карты доступны только по подписке; отсутствует модерация контента~--- множество дублирующихся маршрутов; закрытый исходный код.

\textbf{Google My Maps}~--- бесплатный сервис от Google для создания пользовательских карт~\cite{googlemymaps}. Позволяет ставить маркеры (пины) с фотографиями на карте. Ограничения: не позволяет строить маршрут между маркерами; отсутствуют социальные функции (комментарии, лайки, рейтинги); нет серверной обработки фотографий (компрессия, миниатюры); не предназначен специально для туристических маршрутов.

\textbf{Strava}~--- платформа для записи и анализа спортивной активности~\cite{strava}. Ориентирована на бег и велоспорт, предоставляет детальную статистику тренировок. Ограничения: не предназначена для создания маршрутов с фотографиями~--- фокус на записи спортивных активностей; нет ручного размещения фотографий на карте; нет интерактивного шаринга маршрутов с фото-попапами.

\textbf{Komoot}~--- сервис планирования маршрутов для пешего, велосипедного и горного туризма~\cite{komoot}. Предоставляет навигацию с голосовыми подсказками и информацию о типах покрытия. Ограничения: платные региональные карты; фотографии являются второстепенной функцией~--- нет интерактивного шаринга с попапами фотографий на карте.

\textbf{AllTrails}~--- крупнейший каталог туристических маршрутов с аудиторией более 45~млн пользователей~\cite{alltrails}. Предоставляет обзоры, рейтинги и офлайн-карты. Ограничения: полный функционал доступен только по платной подписке; слабое покрытие территории Российской Федерации; фотографии не привязаны к конкретным точкам маршрута на карте.

Результаты сравнительного анализа представлены в таблице~\ref{tab:competitors}.

\begin{longtable}{|p{4.5cm}|c|c|c|c|c|c|}
\caption{Сравнительный анализ конкурентов}
\label{tab:competitors} \\
\hline
\textbf{Критерий} & \rotatebox{90}{\textbf{Wikiloc}} & \rotatebox{90}{\textbf{Google My Maps}} & \rotatebox{90}{\textbf{Strava}} & \rotatebox{90}{\textbf{Komoot}} & \rotatebox{90}{\textbf{AllTrails}} & \rotatebox{90}{\textbf{Guide Helper}} \\
\hline
\endfirsthead
\hline
\textbf{Критерий} & \rotatebox{90}{\textbf{Wikiloc}} & \rotatebox{90}{\textbf{Google My Maps}} & \rotatebox{90}{\textbf{Strava}} & \rotatebox{90}{\textbf{Komoot}} & \rotatebox{90}{\textbf{AllTrails}} & \rotatebox{90}{\textbf{Guide Helper}} \\
\hline
\endhead
Ручная привязка фото к точкам & Нет & Да & Нет & Частично & Нет & \textbf{Да} \\
\hline
Маршрут между фототочками & Нет & Нет & Нет & Да & Нет & \textbf{Да} \\
\hline
Интерактивный шаринг с фото & Нет & Частично & Нет & Частично & Нет & \textbf{Да} \\
\hline
Бесплатный полный функционал & Нет & Да & Нет & Нет & Нет & \textbf{Да} \\
\hline
Комментарии, лайки, рейтинги & Частично & Нет & Да & Да & Да & \textbf{Да} \\
\hline
Импорт/экспорт GPX, KML, GeoJSON & 2 & 1 & 2 & 2 & 2 & \textbf{3} \\
\hline
Серверная обработка фото & Нет & Нет & Нет & Нет & Нет & \textbf{Да} \\
\hline
Модерация контента & Нет & Нет & Нет & Да & Да & \textbf{Да} \\
\hline
Open-source / self-hosted & Нет & Нет & Нет & Нет & Нет & \textbf{Да} \\
\hline
Профиль высот & Да & Нет & Да & Да & Да & \textbf{Да} \\
\hline
Автоклассификация сложности & Нет & Нет & Нет & Частично & Да & \textbf{Да} \\
\hline
AI-генерация описания & Нет & Нет & Нет & Нет & Нет & \textbf{Да} \\
\hline
AI-ассистент & Нет & Нет & Нет & Нет & Нет & \textbf{Да} \\
\hline
QR-код для шаринга & Нет & Нет & Нет & Нет & Нет & \textbf{Да} \\
\hline
Встраиваемый виджет & Нет & Да & Нет & Нет & Нет & \textbf{Да} \\
\hline
Интеграция с погодой & Нет & Нет & Нет & Да & Да & \textbf{Да} \\
\hline
Режим прохождения маршрута & Нет & Нет & Нет & Нет & Нет & \textbf{Да} \\
\hline
Сезонность маршрутов & Нет & Нет & Нет & Нет & Частично & \textbf{Да} \\
\hline
\end{longtable}

Проведённый анализ показывает, что ни одно из существующих решений не предоставляет полного набора функций, необходимых целевой аудитории. Ключевыми уникальными преимуществами разрабатываемой системы Guide Helper являются:
\begin{itemize}
  \item ручная привязка фотографий к произвольным точкам маршрута на интерактивной карте;
  \item построение маршрута между фототочками с расчётом статистик;
  \item интерактивный шаринг маршрута по ссылке с фото-попапами;
  \item серверная обработка фотографий (компрессия, генерация миниатюр);
  \item AI-функции: генерация описаний маршрутов по фотографиям и AI-ассистент;
  \item полностью бесплатный функционал и открытый исходный код.
\end{itemize}


\section{Анализ целевой аудитории и требования}

На основе анализа предметной области выделены четыре основные группы целевой аудитории системы.

\textbf{Гиды и инструкторы} (основная целевая аудитория)~--- создают маршруты с фотографиями для демонстрации клиентам. Хотят показать, что ждёт участников похода: виды, достопримечательности, сложные участки. Нуждаются в возможности поделиться интерактивной картой по ссылке.

\textbf{Организаторы походов и турклубы}~--- делятся маршрутами с участниками группы до и после похода. Собирают обратную связь через комментарии и рейтинги.

\textbf{Активные путешественники}~--- документируют маршруты с фотографиями для личного использования и обмена опытом с единомышленниками.

\textbf{Туристические компании}~--- используют платформу как витрину маршрутов с фотографиями для привлечения клиентов.

На основе анализа потребностей целевой аудитории сформулированы основные функциональные требования к системе:
\begin{enumerate}
  \item регистрация и аутентификация пользователей с JWT-токенами;
  \item создание, редактирование и удаление маршрутов с точками на карте;
  \item прикрепление фотографий к точкам маршрута с извлечением GPS-координат из EXIF;
  \item асинхронная серверная обработка фотографий (компрессия, генерация миниатюр);
  \item обмен маршрутами по уникальной ссылке с интерактивной картой;
  \item социальные функции: комментарии, лайки, рейтинги маршрутов;
  \item импорт маршрутов из GeoJSON и экспорт в GPX/KML;
  \item расчёт статистик маршрута: расстояние, набор/сброс высоты, время в пути;
  \item интеграция с прогнозом погоды для района маршрута;
  \item AI-генерация описаний маршрутов и AI-ассистент.
\end{enumerate}

Нефункциональные требования включают: время отклика менее 500~мс (95-й перцентиль), обработка фотографий менее 10~с, доступность 99\%, поддержка двух языков (русский, английский), кроссплатформенность (веб-приложение PWA и десктоп Tauri).


\section{Обоснование выбора технологий}

\subsection{Язык программирования серверной части}

Для реализации основных микросервисов (аутентификация, маршруты, обработка фотографий) был выбран язык программирования Rust. Обоснование выбора:

\begin{itemize}
  \item \textbf{Производительность.} Rust компилируется в нативный код без сборщика мусора, что обеспечивает предсказуемую задержку (latency) и минимальное потребление оперативной памяти~--- критично для обработки изображений и работы с WebSocket-соединениями~\cite{rust}.
  \item \textbf{Безопасность памяти.} Система владения (ownership) и проверки заимствований (borrow checker) на этапе компиляции исключает целые классы ошибок: use-after-free, data races, buffer overflow~\cite{rustbook}.
  \item \textbf{Асинхронность.} Экосистема Tokio предоставляет высокопроизводительный асинхронный runtime для обработки тысяч одновременных соединений.
  \item \textbf{Экосистема.} Фреймворк Axum обеспечивает эргономичный API для создания HTTP-сервисов, библиотека sqlx~--- компилируемые SQL-запросы с проверкой типов.
\end{itemize}

Для вспомогательных сервисов (кэширование тайлов, проксирование) выбран язык Go~--- прагматичный выбор для I/O-интенсивных задач с простым конкурентным программированием через горутины~\cite{golang}.

\subsection{Клиентская часть}

Для разработки клиентской части выбран фреймворк React~19~\cite{react} с языком TypeScript. Обоснование:
\begin{itemize}
  \item развитая экосистема компонентов для работы с картами (React Leaflet);
  \item типизация TypeScript снижает количество ошибок на этапе разработки;
  \item поддержка PWA для оффлайн-режима;
  \item возможность обёртки в десктоп-приложение через Tauri.
\end{itemize}

\subsection{Система управления базами данных}

Для хранения данных выбрана СУБД PostgreSQL~16~\cite{postgresql}. Преимущества:
\begin{itemize}
  \item поддержка типа JSONB для хранения гибких структур (точки маршрута с метаданными);
  \item расширение PostGIS для геопространственных запросов;
  \item надёжность и масштабируемость;
  \item богатый набор индексов (B-tree, GIN для JSONB, GiST для геоданных).
\end{itemize}

\subsection{Архитектурный подход: микросервисы}

Для системы выбрана микросервисная архитектура вместо монолитной. Обоснование:
\begin{itemize}
  \item \textbf{Независимое масштабирование.} Сервис обработки фотографий требует больше вычислительных ресурсов, чем сервис аутентификации~--- микросервисы позволяют масштабировать каждый компонент отдельно.
  \item \textbf{Изоляция отказов.} Сбой в обработке фотографий не затрагивает работу аутентификации и маршрутов.
  \item \textbf{Технологическая гибкость.} Возможность использовать Rust для CPU-интенсивных задач и Go для I/O-интенсивных.
  \item \textbf{Независимое развёртывание.} Обновление одного сервиса не требует пересборки остальных.
\end{itemize}

\subsection{Брокер сообщений}

Для асинхронного взаимодействия микросервисов выбран NATS JetStream~\cite{nats}. Сравнение с альтернативами:
\begin{itemize}
  \item \textbf{NATS vs RabbitMQ.} NATS проще в настройке и развёртывании, имеет минимальные требования к ресурсам (один бинарный файл). RabbitMQ требует Erlang runtime и более сложен в администрировании.
  \item \textbf{NATS vs Apache Kafka.} Kafka ориентирован на обработку больших объёмов потоковых данных и избыточен для задач системы. NATS JetStream обеспечивает гарантию доставки <<at-least-once>> при значительно меньшем потреблении ресурсов.
\end{itemize}


\section{Обзор алгоритмов}

\subsection{Формула Гаверсинуса}

Для расчёта расстояния между двумя точками маршрута по их географическим координатам используется формула Гаверсинуса (Haversine formula)~\cite{haversine}. Эта формула вычисляет расстояние по дуге большого круга между двумя точками на сфере по их широте и долготе.

Для двух точек с координатами $(\varphi_1, \lambda_1)$ и $(\varphi_2, \lambda_2)$, где $\varphi$~--- широта, $\lambda$~--- долгота (в радианах):

\begin{equation}
  a = \sin^2\!\left(\frac{\Delta\varphi}{2}\right) + \cos\varphi_1 \cdot \cos\varphi_2 \cdot \sin^2\!\left(\frac{\Delta\lambda}{2}\right)
\end{equation}

\begin{equation}
  c = 2 \cdot \mathrm{atan2}\!\left(\sqrt{a},\;\sqrt{1 - a}\right)
\end{equation}

\begin{equation}
  d = R \cdot c
\end{equation}

где $\Delta\varphi = \varphi_2 - \varphi_1$, $\Delta\lambda = \lambda_2 - \lambda_1$, $R = 6{,}371$~км~--- средний радиус Земли.

Общее расстояние маршрута вычисляется как сумма расстояний между последовательными точками:
\begin{equation}
  D = \sum_{i=1}^{n-1} d(\text{точка}_i, \text{точка}_{i+1})
\end{equation}

Формула Гаверсинуса обеспечивает точность порядка 0,5\% для расстояний менее 10\,000~км, что достаточно для туристических маршрутов.

\subsection{Формула Нейсмита}

Для оценки времени прохождения пешего маршрута используется формула Нейсмита (Naismith's rule)~\cite{naismith}, разработанная шотландским альпинистом Уильямом Нейсмитом в 1892~году. Правило устанавливает, что пешеход проходит 5~км в час по ровной местности, и дополнительно требуется 1~час на каждые 600~м набора высоты:

\begin{equation}
  T = \frac{D}{V} + \frac{H^{+}}{600}
\end{equation}

где $T$~--- время в часах, $D$~--- горизонтальное расстояние в километрах, $V = 5$~км/ч~--- скорость на ровной местности, $H^{+}$~--- суммарный набор высоты в метрах.

Набор высоты вычисляется по данным высот точек маршрута, полученным через Open-Meteo Elevation API:
\begin{equation}
  H^{+} = \sum_{i=1}^{n-1} \max(0,\; h_{i+1} - h_i)
\end{equation}

где $h_i$~--- высота $i$-й точки маршрута над уровнем моря.

\subsection{Алгоритм классификации сложности маршрута}

Для автоматической классификации сложности маршрута разработан алгоритм на основе комбинации метрик расстояния и набора высоты. Алгоритм присваивает маршруту одну из трёх категорий сложности: <<Лёгкий>>, <<Средний>>, <<Сложный>>.

Входные параметры: общее расстояние $D$ (км), суммарный набор высоты $H^{+}$ (м). Каждый параметр оценивается по балльной шкале:

\begin{itemize}
  \item Расстояние: $D < 10$~км~--- 1 балл, $10 \leq D < 20$~км~--- 2 балла, $D \geq 20$~км~--- 3 балла.
  \item Набор высоты: $H^{+} < 500$~м~--- 1 балл, $500 \leq H^{+} < 1000$~м~--- 2 балла, $H^{+} \geq 1000$~м~--- 3 балла.
\end{itemize}

Итоговый балл $S = S_D + S_{H^{+}}$ определяет категорию: $S \leq 2$~--- <<Лёгкий>>, $3 \leq S \leq 4$~--- <<Средний>>, $S \geq 5$~--- <<Сложный>>.

\subsection{Алгоритм компрессии изображений}

Для оптимизации хранения и передачи фотографий реализован алгоритм серверной обработки изображений, выполняемый асинхронно через брокер сообщений NATS. Алгоритм состоит из следующих этапов:

\begin{enumerate}
  \item Получение сообщения из очереди NATS с метаданными загруженной фотографии.
  \item Декодирование изображения из формата JPEG/PNG.
  \item Масштабирование: если ширина изображения превышает $W_{\max} = 1920$~пикселей, выполняется пропорциональное уменьшение с использованием фильтра Ланцоша (Lanczos3) для сохранения качества.
  \item Генерация миниатюры (thumbnail): масштабирование до ширины $W_{th} = 300$~пикселей для быстрой загрузки превью на карте.
  \item Кодирование в JPEG с качеством $Q = 85\%$~--- баланс между размером файла и визуальным качеством.
  \item Загрузка обработанных изображений в объектное хранилище MinIO (S3-совместимый API).
  \item Обновление записи в базе данных: сохранение URL оригинала и миниатюры, установка статуса <<обработано>>.
\end{enumerate}


\section{Проектирование архитектуры}

\subsection{Диаграмма контекста системы}

На рисунке~\ref{fig:c4-context} представлена диаграмма контекста (C4 Context) системы Guide Helper, показывающая систему в окружении внешних пользователей и сервисов.

Система взаимодействует с четырьмя внешними сервисами:
\begin{itemize}
  \item \textbf{OpenStreetMap Tile API}~--- предоставляет картографические тайлы для отображения интерактивной карты;
  \item \textbf{Open-Meteo API}~--- предоставляет данные о высоте рельефа и прогноз погоды;
  \item \textbf{Nominatim}~--- сервис геокодирования для поиска местоположений по названию;
  \item \textbf{Ollama}~--- локальный сервер для запуска LLM-модели (llama3.2-vision) для AI-функций.
\end{itemize}

\begin{figure}[H]
  \centering
  \includegraphics[width=0.95\textwidth]{images/c4-context.png}
  \caption{Диаграмма контекста системы (C4 Context)}
  \label{fig:c4-context}
\end{figure}

\subsection{Диаграмма контейнеров}

На рисунке~\ref{fig:c4-container} представлена диаграмма контейнеров (C4 Container), детализирующая внутреннюю структуру системы.

Система состоит из следующих контейнеров:
\begin{itemize}
  \item \textbf{Frontend}~--- SPA на React~19 с TypeScript, интерактивные карты на Leaflet;
  \item \textbf{Auth Service}~--- микросервис аутентификации на Rust (Axum), JWT-токены, Argon2;
  \item \textbf{Routes Service}~--- микросервис управления маршрутами на Rust (Axum), REST API, WebSocket;
  \item \textbf{Photo Worker}~--- микросервис обработки фотографий на Rust, потребитель NATS;
  \item \textbf{Cache Service}~--- микросервис кэширования тайлов на Go (Gin), Redis;
  \item \textbf{Tiles Service}~--- микросервис проксирования тайлов на Go (Gin);
  \item \textbf{PostgreSQL}~--- СУБД (две базы: auth\_db и routes\_db);
  \item \textbf{Redis}~--- кэш тайлов карт;
  \item \textbf{NATS JetStream}~--- брокер сообщений для асинхронной обработки фото;
  \item \textbf{MinIO}~--- S3-совместимое объектное хранилище для фотографий.
\end{itemize}

\begin{figure}[H]
  \centering
  \includegraphics[width=0.95\textwidth]{images/c4-container.png}
  \caption{Диаграмма контейнеров системы (C4 Container)}
  \label{fig:c4-container}
\end{figure}

\subsection{Диаграмма последовательности обработки фотографий}

На рисунке~\ref{fig:sequence-photo} представлена диаграмма последовательности (Sequence Diagram), описывающая процесс асинхронной обработки фотографий~--- одного из ключевых сценариев работы системы.

Процесс состоит из следующих шагов:
\begin{enumerate}
  \item Пользователь загружает фотографию через веб-интерфейс.
  \item Frontend извлекает GPS-координаты из EXIF-метаданных фотографии.
  \item Frontend отправляет фотографию (base64) и координаты на Routes API.
  \item Routes Service сохраняет точку маршрута с фотографией в статусе <<pending>> в PostgreSQL.
  \item Routes Service публикует сообщение в NATS JetStream с метаданными для обработки.
  \item Photo Worker получает сообщение из очереди NATS.
  \item Photo Worker выполняет обработку: компрессию, масштабирование, генерацию миниатюры.
  \item Photo Worker загружает обработанные файлы в MinIO.
  \item Photo Worker обновляет запись в PostgreSQL: URL файлов и статус <<done>>.
  \item Routes Service отправляет уведомление через WebSocket на Frontend.
  \item Frontend обновляет отображение фотографии на карте.
\end{enumerate}

\begin{figure}[H]
  \centering
  \includegraphics[width=0.95\textwidth]{images/sequence-photo.png}
  \caption{Диаграмма последовательности обработки фотографий}
  \label{fig:sequence-photo}
\end{figure}

\subsection{Проектирование базы данных}

Система использует две отдельные базы данных PostgreSQL для обеспечения изоляции сервисов.

\textbf{База данных auth\_db} содержит таблицу \texttt{users} для хранения учётных данных пользователей: идентификатор (UUID), email, хеш пароля (Argon2), имя, URL аватара, временные метки создания и обновления, поддержка мягкого удаления (soft delete).

\textbf{База данных routes\_db} содержит следующие таблицы:
\begin{itemize}
  \item \texttt{routes}~--- маршруты (UUID, user\_id, название, точки в формате JSONB, токен для шаринга);
  \item \texttt{comments}~--- комментарии к маршрутам;
  \item \texttt{route\_likes}~--- лайки маршрутов (уникальная пара route\_id + user\_id);
  \item \texttt{route\_ratings}~--- рейтинги маршрутов (1--5 баллов, уникальная пара route\_id + user\_id).
\end{itemize}

Точки маршрута хранятся в поле \texttt{points} типа JSONB в таблице \texttt{routes}. Каждая точка содержит: идентификатор, координаты (широта, долгота) и опциональные данные фотографии (URL оригинала, URL миниатюры, статус обработки).

ER-диаграмма базы данных представлена на рисунке~\ref{fig:er-diagram}.

\begin{figure}[H]
  \centering
  \includegraphics[width=0.85\textwidth]{images/er-diagram.png}
  \caption{ER-диаграмма базы данных}
  \label{fig:er-diagram}
\end{figure}


\section{Макеты страниц}

В данном разделе представлены макеты (скриншоты) основных страниц работающего веб-приложения Guide Helper.

На рисунке~\ref{fig:mockup-map} представлена главная страница~--- редактор маршрутов с интерактивной картой. Пользователь может добавлять точки на карту, прикреплять к ним фотографии, строить маршрут между точками автоматически (через OSRM) или вручную. На правой панели отображаются статистики маршрута: расстояние, набор и сброс высоты, расчётное время в пути.

\begin{figure}[H]
  \centering
  \includegraphics[width=0.95\textwidth]{images/mockups/map-editor.png}
  \caption{Редактор маршрутов с интерактивной картой}
  \label{fig:mockup-map}
\end{figure}

На рисунке~\ref{fig:mockup-shared} представлена страница публичного просмотра маршрута, доступная по уникальной ссылке. Отображается карта с маршрутом и фотографиями, комментарии, лайки и рейтинг.

\begin{figure}[H]
  \centering
  \includegraphics[width=0.95\textwidth]{images/mockups/shared-route.png}
  \caption{Страница публичного просмотра маршрута}
  \label{fig:mockup-shared}
\end{figure}

На рисунке~\ref{fig:mockup-profile} представлена страница профиля пользователя с вкладками: профиль, безопасность, список маршрутов. Пользователь может управлять своими маршрутами: просматривать, редактировать, удалять, экспортировать.

\begin{figure}[H]
  \centering
  \includegraphics[width=0.95\textwidth]{images/mockups/profile.png}
  \caption{Страница профиля пользователя}
  \label{fig:mockup-profile}
\end{figure}
